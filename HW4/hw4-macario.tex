\documentclass[12pt]{article}
\usepackage[utf8]{inputenc}

\usepackage{hyperref}   % Change color and style of \ref
\hypersetup{
    colorlinks=true,
    linkcolor=blue,
    filecolor=magenta,
    urlcolor=cyan,
    pdftitle={Overleaf Example},
    pdfpagemode=FullScreen,
}

\usepackage{graphicx} % Allows you to insert figures
\graphicspath{ {./images/} } % images are found in this position
\counterwithin{figure}{section} % Settings for figure numbering
\counterwithin{figure}{subsection}

\usepackage{amsmath} % Allows you to do equations
\usepackage{amsfonts} % Contains math. symbols fonts (e.g., 'real numbers' set)
\usepackage{fancyhdr} % Formats the header
\usepackage{geometry} % Formats the paper size, orientation, and margins
\linespread{1.25} % about 1.5 spacing in Word
\setlength{\parindent}{0pt} % no paragraph indents
\setlength{\parskip}{1em} % paragraphs separated by one line

\usepackage{enumitem} % Used to reduce whitespace between list elements
\setlist[itemize]{noitemsep, topsep=0pt} % Set the whitespace above list to the minimum

\usepackage[style=authoryear-ibid,backend=biber,maxbibnames=99,maxcitenames=2,uniquelist=false,isbn=false,url=true,eprint=false,doi=true,giveninits=true,uniquename=init]{biblatex} % Allows you to do citations - does Harvard style and compatible with Zotero
\urlstyle{same} % makes a nicer URL and DOI font
\AtEveryBibitem{
    \clearfield{urlyear}
    \clearfield{urlmonth}
} % removes access date
\AtEveryBibitem{\clearfield{month}} % removes months in bibliography
\AtEveryCitekey{\clearfield{month}} % removes months in citations
\renewbibmacro{in:}{} % Removes the 'In' before journal names

\renewbibmacro*{editorstrg}{%from biblatex.def
  \printtext[editortype]{%
    \iffieldundef{editortype}
      {\ifboolexpr{
        test {\ifnumgreater{\value{editor}}{1}}
        or
        test {\ifandothers{editor}}
        }
        {\bibcpstring{editors}}
        {\bibcpstring{editor}}}
    {\ifbibxstring{\thefield{editortype}}
        {\ifboolexpr{
            test {\ifnumgreater{\value{editor}}{1}}
            or
            test {\ifandothers{editor}}
            }
            {\bibcpstring{\thefield{editortype}s}}%changed
          {\bibcpstring{\thefield{editortype}}}}%changed
        {\thefield{editortype}}}}}

\renewbibmacro*{byeditor+others}{%from biblatex.def
  \ifnameundef{editor}
    {}
    {\printnames[byeditor]{editor}%
     \addspace%added
     \mkbibparens{\usebibmacro{editorstrg}}%added
     \clearname{editor}%
     \newunit}%
  \usebibmacro{byeditorx}%
  \usebibmacro{bytranslator+others}}
  % The commands above from lines 20-49 change the way editors are displayed in books
\AtEveryBibitem{%
  \clearlist{language}%
} % removes language from bibliography
% Removes ibids (ibidems)
\DeclareNameAlias{sortname}{family-given} % Ensures the names of the authors after the first author are in the correct order in the bibliography
\renewcommand*{\revsdnamepunct}{} % Corrects punctuation for authors with just a first initial
%\addbibresource{Example.bib} % Tells LaTeX where the citations are coming from. This is imported from Zotero
\usepackage[format=plain,
            font=it]{caption} % Italicizes figure captions
\usepackage[english]{babel}
\usepackage{csquotes}
\renewcommand*{\nameyeardelim}{\addcomma\space} % Adds comma in in-text citations
\renewcommand{\headrulewidth}{0pt}
\geometry{letterpaper, portrait, margin=1in}
\setlength{\headheight}{14.49998pt}

\newcommand\titleofdoc{Homework 4 – Backpropagation algorithm} %%%%% Title
\newcommand\GroupName{Davide Macario}
\newcommand\CurrDate{October 3\textsuperscript{rd} 2023}

\begin{document}
\begin{titlepage}
   \begin{center}
        \vspace*{4cm} % Adjust spacings to ensure the title page is generally filled with text

        \Huge{\titleofdoc}

        \vspace{0.5cm}
        \LARGE{ECE 559 – Neural Networks}

        \vspace{3 cm}
        \Large{\GroupName\\ }
        \large{UIN:\@ 660603047}


        \vspace{2 cm} % Optional additional info here


        \vspace{3 cm}
        \Large{\CurrDate}

        \vspace{0.25 cm}
        \Large{Fall 2023}


        \vfill
    \end{center}
\end{titlepage}

\setcounter{page}{2}
\pagestyle{fancy}
\fancyhf{}
\rhead{\thepage}
\lhead{\GroupName; \titleofdoc}

\section{Introduction}

This homework activity consisted of the Python implementation of the backpropagation (BP) algorithm for training a neural network composed of two layers, to approximate a function locally.
The training set was composed of $n=300$ random points having $x$ uniformly-distributed in $[0,\ 1]$, and the function to be approximated was $d_i = f(x_i) = \sin{(20x_i) + 3x_i + \nu_i}$, where $\nu_i$ is random uniform noise in the range $[-0.1,\ 0.1]$, of which $n$ samples have been extracted as well.

The considered neural network is composed of two neuron layers, takes a single input, and returns a single output. 
The central layer is composed of $N=24$ neurons using as activation function the $\tanh{v}$ function, while the output layer has a linear activation function ($\phi(v) = v$).
As a result, the total number of parameters that identify the model, including biases, is $3N+1 = 73$.

\section{Backpropagation algorithm}\label{sec:bp}

The backpropagation algorithm aims at finding the best model parameters to minimize the output Mean Square Error (MSE), which is defined as, for the training set $\mathcal{S}$:

\begin{equation}
  \mathcal{E} = \frac{1}{|\mathcal{S}|} \sum_{i=0}^{|\mathcal{S}|} {|| d_i - f(x_i, \textbf{w}) ||^2}
\end{equation}
where $(x_i, d_i)$ are the training set points in this case, and $f(x_i, \textbf{w})$ corresponds to the neural network output when feeding $x_i$ as input, using weights $\textbf{w}$.

The training procedure exploits the gradient descent algorithm, by correcting the weights by the gradient of $\mathcal{E}$, after evaluating it for each element of the training set.
This approach, in which the weights are updated once for each 

\end{document}