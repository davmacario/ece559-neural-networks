\documentclass[12pt]{article}
\usepackage[utf8]{inputenc}

\usepackage{hyperref}   % Change color and style of \ref
\hypersetup{
    colorlinks=true,
    linkcolor=blue,
    filecolor=magenta,
    urlcolor=cyan,
    pdftitle={Overleaf Example},
    pdfpagemode=FullScreen,
}

\usepackage{graphicx} % Allows you to insert figures
\graphicspath{ {./img/} } % images are found in this position
\counterwithin{figure}{section} % Settings for figure numbering
\counterwithin{figure}{subsection}

\usepackage{amsmath} % Allows you to do equations
\usepackage{amsfonts} % Contains math. symbols fonts (e.g., 'real numbers' set)
\usepackage{fancyhdr} % Formats the header
\usepackage{geometry} % Formats the paper size, orientation, and margins
\linespread{1.25} % about 1.5 spacing in Word
\setlength{\parindent}{0pt} % no paragraph indents
\setlength{\parskip}{1em} % paragraphs separated by one line

\usepackage{enumitem} % Used to reduce whitespace between list elements
\setlist[itemize]{noitemsep, topsep=0pt} % Set the whitespace above list to the minimum

\usepackage[style=authoryear-ibid,backend=biber,maxbibnames=99,maxcitenames=2,uniquelist=false,isbn=false,url=true,eprint=false,doi=true,giveninits=true,uniquename=init]{biblatex} % Allows you to do citations - does Harvard style and compatible with Zotero
\urlstyle{same} % makes a nicer URL and DOI font
\AtEveryBibitem{
    \clearfield{urlyear}
    \clearfield{urlmonth}
} % removes access date
\AtEveryBibitem{\clearfield{month}} % removes months in bibliography
\AtEveryCitekey{\clearfield{month}} % removes months in citations
\renewbibmacro{in:}{} % Removes the 'In' before journal names

\renewbibmacro*{editorstrg}{%from biblatex.def
  \printtext[editortype]{%
    \iffieldundef{editortype}
      {\ifboolexpr{
        test {\ifnumgreater{\value{editor}}{1}}
        or
        test {\ifandothers{editor}}
        }
        {\bibcpstring{editors}}
        {\bibcpstring{editor}}}
    {\ifbibxstring{\thefield{editortype}}
        {\ifboolexpr{
            test {\ifnumgreater{\value{editor}}{1}}
            or
            test {\ifandothers{editor}}
            }
            {\bibcpstring{\thefield{editortype}s}}%changed
          {\bibcpstring{\thefield{editortype}}}}%changed
        {\thefield{editortype}}}}}

\renewbibmacro*{byeditor+others}{%from biblatex.def
  \ifnameundef{editor}
    {}
    {\printnames[byeditor]{editor}%
     \addspace%added
     \mkbibparens{\usebibmacro{editorstrg}}%added
     \clearname{editor}%
     \newunit}%
  \usebibmacro{byeditorx}%
  \usebibmacro{bytranslator+others}}
  % The commands above from lines 20-49 change the way editors are displayed in books
\AtEveryBibitem{%
  \clearlist{language}%
} % removes language from bibliography
% Removes ibids (ibidems)
\DeclareNameAlias{sortname}{family-given} % Ensures the names of the authors after the first author are in the correct order in the bibliography
\renewcommand*{\revsdnamepunct}{} % Corrects punctuation for authors with just a first initial
%\addbibresource{Example.bib} % Tells LaTeX where the citations are coming from. This is imported from Zotero
\usepackage[format=plain,
            font=it]{caption} % Italicizes figure captions
\usepackage[english]{babel}
\usepackage{csquotes}
\renewcommand*{\nameyeardelim}{\addcomma\space} % Adds comma in in-text citations
\renewcommand{\headrulewidth}{0pt}
\geometry{letterpaper, portrait, margin=1in}
\setlength{\headheight}{14.49998pt}

\newcommand\titleofdoc{Homework 2 - Perceptron Learning Algorithm} %%%%% Title
\newcommand\GroupName{Davide Macario}
\newcommand\CurrDate{September 19\textsuperscript{th} 2023}

\begin{document}
\begin{titlepage}
   \begin{center}
        \vspace*{4cm} % Adjust spacings to ensure the title page is generally filled with text

        \Huge{\titleofdoc}

        \vspace{0.5cm}
        \LARGE{ECE 559 – Neural Networks}

        \vspace{3 cm}
        \Large{\GroupName\\ }


        \vspace{2 cm} % Optional additional info here


        \vspace{3 cm}
        \Large{\CurrDate}

        \vspace{0.25 cm}
        \Large{Fall 2023}


        \vfill
    \end{center}
\end{titlepage}

\setcounter{page}{2}
\pagestyle{fancy}
\fancyhf{}
\rhead{\thepage}
\lhead{\GroupName; \titleofdoc}

\section{Introduction}

This homework activity consists in the implementation and analysis of the Perceptron Learning Algorithm implemented for a single-neuron neural network, using the step activation function.
The analysis is based on changing the learning rate $\eta$ and the number of training elements $n$.

The considered neuron has 2 inputs $x_1$ and $x_2$, while the parameters are $w_0$, $w_1$ and $w_2$.
As a result, the function implemented by the perceptron is:

\begin{equation}
    y = u(w_0 + w_1 x_1 + w_2 x_2)
\end{equation}

Corresponding to the linear boundary $w_0 + w_1 x_1 + w_2 x_2 = 0$.

The experiments have been conducted by taking as actual weights the values $w_0 = 0.1146$, $w_1 = -0.4883$ and $w_2 = -0.3315$.
The randomly initialized values of the weights, instead, have been set to: $w_0' = 0.2184$, $w_1' = -0.4940$, $w_2' = -0.5434$.

\section{Comments on the results}

The system was initially analyzed by providing a training set $\mathcal{S}$ containing $n = 100$ elements.
For $\eta = 1$, the PLA converged in 9 epoch.
Figure \ref{fig:01} shows the behavior of the number of misclassified training elements.

\begin{image}



















    

\end{document}