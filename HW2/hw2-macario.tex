\documentclass[12pt]{article}
\usepackage[utf8]{inputenc}

\usepackage{hyperref}   % Change color and style of \ref
\hypersetup{
    colorlinks=true,
    linkcolor=blue,
    filecolor=magenta,
    urlcolor=cyan,
    pdftitle={Overleaf Example},
    pdfpagemode=FullScreen,
}

\usepackage{graphicx} % Allows you to insert figures
\graphicspath{ {./img/} } % images are found in this position
\counterwithin{figure}{section} % Settings for figure numbering
\counterwithin{figure}{subsection}

\usepackage{amsmath} % Allows you to do equations
\usepackage{amsfonts} % Contains math. symbols fonts (e.g., 'real numbers' set)
\usepackage{fancyhdr} % Formats the header
\usepackage{geometry} % Formats the paper size, orientation, and margins
\linespread{1.25} % about 1.5 spacing in Word
\setlength{\parindent}{0pt} % no paragraph indents
\setlength{\parskip}{1em} % paragraphs separated by one line

\usepackage{enumitem} % Used to reduce whitespace between list elements
\setlist[itemize]{noitemsep, topsep=0pt} % Set the whitespace above list to the minimum

\usepackage[style=authoryear-ibid,backend=biber,maxbibnames=99,maxcitenames=2,uniquelist=false,isbn=false,url=true,eprint=false,doi=true,giveninits=true,uniquename=init]{biblatex} % Allows you to do citations - does Harvard style and compatible with Zotero
\urlstyle{same} % makes a nicer URL and DOI font
\AtEveryBibitem{
    \clearfield{urlyear}
    \clearfield{urlmonth}
} % removes access date
\AtEveryBibitem{\clearfield{month}} % removes months in bibliography
\AtEveryCitekey{\clearfield{month}} % removes months in citations
\renewbibmacro{in:}{} % Removes the 'In' before journal names

\renewbibmacro*{editorstrg}{%from biblatex.def
  \printtext[editortype]{%
    \iffieldundef{editortype}
      {\ifboolexpr{
        test {\ifnumgreater{\value{editor}}{1}}
        or
        test {\ifandothers{editor}}
        }
        {\bibcpstring{editors}}
        {\bibcpstring{editor}}}
    {\ifbibxstring{\thefield{editortype}}
        {\ifboolexpr{
            test {\ifnumgreater{\value{editor}}{1}}
            or
            test {\ifandothers{editor}}
            }
            {\bibcpstring{\thefield{editortype}s}}%changed
          {\bibcpstring{\thefield{editortype}}}}%changed
        {\thefield{editortype}}}}}

\renewbibmacro*{byeditor+others}{%from biblatex.def
  \ifnameundef{editor}
    {}
    {\printnames[byeditor]{editor}%
     \addspace%added
     \mkbibparens{\usebibmacro{editorstrg}}%added
     \clearname{editor}%
     \newunit}%
  \usebibmacro{byeditorx}%
  \usebibmacro{bytranslator+others}}
  % The commands above from lines 20-49 change the way editors are displayed in books
\AtEveryBibitem{%
  \clearlist{language}%
} % removes language from bibliography
% Removes ibids (ibidems)
\DeclareNameAlias{sortname}{family-given} % Ensures the names of the authors after the first author are in the correct order in the bibliography
\renewcommand*{\revsdnamepunct}{} % Corrects punctuation for authors with just a first initial
%\addbibresource{Example.bib} % Tells LaTeX where the citations are coming from. This is imported from Zotero
\usepackage[format=plain,
            font=it]{caption} % Italicizes figure captions
\usepackage[english]{babel}
\usepackage{csquotes}
\renewcommand*{\nameyeardelim}{\addcomma\space} % Adds comma in in-text citations
\renewcommand{\headrulewidth}{0pt}
\geometry{letterpaper, portrait, margin=1in}
\setlength{\headheight}{14.49998pt}

\newcommand\titleofdoc{Homework 2 - Perceptron Learning Algorithm} %%%%% Title
\newcommand\GroupName{Davide Macario}
\newcommand\CurrDate{September 19\textsuperscript{th} 2023}

\begin{document}
\begin{titlepage}
   \begin{center}
        \vspace*{4cm} % Adjust spacings to ensure the title page is generally filled with text

        \Huge{\titleofdoc}

        \vspace{0.5cm}
        \LARGE{ECE 559 – Neural Networks}

        \vspace{3 cm}
        \Large{\GroupName\\ }


        \vspace{2 cm} % Optional additional info here


        \vspace{3 cm}
        \Large{\CurrDate}

        \vspace{0.25 cm}
        \Large{Fall 2023}


        \vfill
    \end{center}
\end{titlepage}

\setcounter{page}{2}
\pagestyle{fancy}
\fancyhf{}
\rhead{\thepage}
\lhead{\GroupName; \titleofdoc}

\section{Introduction}

This homework activity consists in the implementation and analysis of the Perceptron Learning Algorithm implemented for a single-neuron neural network, using the step activation function.
The analysis is based on changing the learning rate $\eta$ and the number of training elements $n$.

The considered neuron has 2 inputs $x_1$ and $x_2$, while the parameters are $w_0$, $w_1$ and $w_2$.
As a result, the function implemented by the perceptron is:

\begin{equation}
    y = u(w_0 + w_1 x_1 + w_2 x_2)
\end{equation}

Corresponding to the linear boundary $w_0 + w_1 x_1 + w_2 x_2 = 0$.

The experiments have been conducted by taking as actual weights the values $w_0 = 0.1146$, $w_1 = -0.4883$ and $w_2 = -0.3315$.
The randomly initialized values of the weights, instead, have been set to: $w_0' = 0.2184$, $w_1' = -0.4940$, $w_2' = -0.5434$.

\section{Comments on the results}

\subsection{Case $n=100$, $\eta=1$}
\label{sec:100-1}

The system was initially analyzed by providing a training set $\mathcal{S}$ containing $n = 100$ elements in ${[-1, 1]}^{2}$.
For $\eta = 1$, the PLA converged in 9 epochs.
Figure \ref{fig:01} displays the $100$ training elements on the $x_1, x_2$ plane and the linear boundary.
Figure \ref{fig:02}, instead shows the behavior of the number of misclassified training elements as a function of the number of training epochs.

\begin{figure} [ht]
    \begin{minipage}{0.45\linewidth}
    \centering
        \includegraphics[width=\textwidth]{img/bound_old_100-1.png}
        \caption{Training set elements and decision boundary}
        \label{fig:01}
    \end{minipage}
    \hspace{0.5cm}
    \begin{minipage}{0.45\linewidth}
    \centering
        \includegraphics[width=\textwidth]{img/miss_epoch_100-1.png}
        \caption{Misclassifications vs.\ epoch; $n=100$, $\eta=1$}
        \label{fig:02}
    \end{minipage}
\end{figure}

Table \ref{tab:01} contains the comparison between the actual weight values and the ones obtained from the PTA.

\begin{table}[h]
    \centering
    \begin{tabular}{|c|c|c|}
        \hline
        Weight & \textbf{Original} & \textbf{PTA} \\ 
        \hline
        \hline
        $w_0$ & 0.1146 & 1.218 \\
        \hline
        $w_1$ & -0.4883 & -5.419 \\
        \hline
        $w_2$ & -0.3315 & -3.599 \\
        \hline
    \end{tabular}
    \caption{Comparison between actual weights and results of PTA}
    \label{tab:01}
\end{table}

It is possible to see that the weights obtained after the learning algorithm are approximately 10 times the actual weights.
Despite being different from the actual weights, these values can provide a correct classification since the decision boundary can be written as:

\begin{equation}
    x_2 = \frac{-w_0 - w_1 x_1}{w_2}
\end{equation}

Therefore, it is enough for the ratio of the weights to remain the same in order to provide correct classification.

By evaluating the average convergence time in epochs over 50 different runs of the algorithm (changing the weights, the training set elements and the randomly–initialized weights at each iteration) for $n=100$ and $\eta=1$, the algorithm has been observed to converge in 65.2 epochs on average.
This highlights how the actual time of convergence depends on how the weights are initialized at the beginning of PTA and on the training set provided.

\subsection{Case $n=100$, $\eta=10$}
\label{sec:100-10}

Next, the experiment was repeated after increasing the learning coefficient to $\eta = 10$.
Figure \ref{fig:100-10-1} reports the relationship between number of epochs and misclassifications on the training set.
In this case, the algorithm took 10 epochs to converge.

\begin{figure}[h]
    \centering
    \includegraphics[width=0.7\linewidth]{img/miss_epoch_100-10.png}
    \caption{Misclassifications vs.\ epoch, $n=100$, $\eta=10$}
    \label{fig:100-10-1}
\end{figure}

The final weights obtained for this run are displayed in table \ref{tab:02}.

\begin{table}[h]
    \centering
    \begin{tabular}{|c|c|c|}
        \hline
        Weight & \textbf{Original} & \textbf{PTA} \\ 
        \hline
        \hline
        $w_0$ & 0.1146 & 10.228 \\
        \hline
        $w_1$ & -0.4883 & -49.42 \\
        \hline
        $w_2$ & -0.3315 & -34.25 \\
        \hline
    \end{tabular}
    \caption{Comparison between actual weights and results of PTA ($n=100$, $\eta=10$)}
    \label{tab:02}
\end{table}

Again, despite the final weights being different from the initial ones, they yield approximately the same boundary equation as the actual weights, achieving correct classification.

Evaluating the performance over 50 runs, the average number of epochs has been found to be 40.

\subsection{Case $n=100$, $\eta=0.1$}
\label{sec:100-01}

Next, the learning coefficient was set to $0.1$.
With this value of $\eta$, the PTA took 12 epochs to converge, as displayed in figure \ref{fig:100-01-1}.

\begin{figure}[h]
    \centering
    \includegraphics[width=0.7\linewidth]{img/miss_epoch_100-0.1.png}
    \caption{Misclassifications vs.\ epoch, $n=100$, $\eta=0.1$}
    \label{fig:100-01-1}
\end{figure}

The final weights are reported in table \ref{tab:03}.
The values are now of the same order of magnitude, but they are still different.
In any case, they yield correct classification, since the ratios $\frac{-w_0}{w_2}$ and $\frac{-w_1}{w_2}$ are close to the original ones.

\begin{table}[h]
    \centering
    \begin{tabular}{|c|c|c|}
        \hline
        Weight & \textbf{Original} & \textbf{PTA} \\ 
        \hline
        \hline
        $w_0$ & 0.1146 & 0.2184 \\
        \hline
        $w_1$ & -0.4883 & -0.8969 \\
        \hline
        $w_2$ & -0.3315 & -0.5892 \\
        \hline
    \end{tabular}
    \caption{Comparison between actual weights and results of PTA ($n=100$, $\eta=0.1$)}
    \label{tab:03}
\end{table}

\subsection{Increasing the training set size}

In general, in supervised learning it is possible to increase the size of the training set to achieve better classification performance.
This happens because having more elements to train on, the model is able to learn the patterns that identify the classes in a much more accurate way.
For this section, the training set size has been increased to $n=1000$, obtaining the elements reported in figure \ref{fig:train-1000}.
By observing the image, it is evident how in this case there is much little room of error for finding the linear decision boundary that separates the two classes.

\begin{figure} [h]
    \begin{minipage}{0.45\linewidth}
    \centering
        \includegraphics[width=\textwidth]{img/bound_old_1000-1.png}
        \caption{Decision boundary and training set elements, $n=1000$}
        \label{fig:train-1000}
    \end{minipage}
    \hspace{0.5cm}
    \begin{minipage}{0.45\linewidth}
    \centering
        \includegraphics[width=\textwidth]{img/miss_epoch_1000-1.png}
        \caption{Misclassifications vs.\ epoch, $\eta=1$}
        \label{fig:miss-1000-1}
    \end{minipage}
\end{figure}

Running the algorithm for $\eta = [1, 10, 0.1]$, the resulting number of misclassifications vs. number of epochs can be seen in figures~\ref{fig:miss-1000-1},~\ref{fig:miss-1000-10} and~\ref{fig:miss-1000-01}, respectively.
For $\eta=1$, the algorithm converged in 338 epochs, for $\eta=10$ in 84 epochs, and for $\eta=0.1$ it took 97 epochs.
As expected, training requires more iterations in this case, since a lower error is allowed.

\begin{figure} [h]
    \begin{minipage}{0.45\linewidth}
    \centering
        \includegraphics[width=\textwidth]{img/miss_epoch_1000-10.png}
        \caption{Misclassifications vs.\ epoch, $\eta=10$}
        \label{fig:miss-1000-10}
    \end{minipage}
    \hspace{0.5cm}
    \begin{minipage}{0.45\linewidth}
    \centering
        \includegraphics[width=\textwidth]{img/miss_epoch_1000-0.1.png}
        \caption{Misclassifications vs.\ epoch, $\eta=0.1$}
        \label{fig:miss-1000-01}
    \end{minipage}
\end{figure}

Table \ref{tab:eta-1000} reports the original weights and the estimated ones for every value of $\eta$.
By evaluating the ratios $\frac{-w_0}{w_2}$ and $\frac{-w_1}{w_2}$, it can be seen how they are closer to the original ones compared to the previous case ($n=100$).

\begin{table}[h]
    \centering
    \begin{tabular}{|c|c|c|c|c|}
        \hline
        Weight & \textbf{Original} & \textbf{PTA}, $\eta=1$ & \textbf{PTA}, $\eta=10$ & \textbf{PTA}, $\eta=0.1$ \\ 
        \hline
        \hline
        $w_0$ & 0.1146 & 11.26 & 60.26 & 0.6566 \\
        \hline
        $w_1$ & -0.4883 & -47.87 & -257.54 & -2.795 \\
        \hline
        $w_2$ & -0.3315 & -32.57 & -175.41 & -1.900 \\
        \hline
    \end{tabular}
    \caption{Comparison between actual weights and results of PTA ($n=1000$)}
    \label{tab:eta-1000}
\end{table}

For each value of the learning coefficient, the training was performed 50 times in order to evaluate the average time of convergence.
For $\eta=1$ the average number of epochs is 338, for $\eta=10$ it is 84, and for $\eta=97$ it is 97.
Once again, these values highlight the dependence of convergence time on the random initialization values.

\end{document}